\section{河南省道路安全条例}

\subsection*{罚100}

\noindent 驾驶机动车有下列情形之一的,处一百元罚款:

\begin{itemize}
    \item 违反分道行驶规定的;
    \item 未按照交通标志、标线指示或者交通警察指挥行驶的;
    \item 违反倒车规定的;
    \item 违反牵引挂车规定的;
    \item 违反依次交替通行规定的;
    \item 违反试车规定的;
    \item 违反灯光使用规定的;
    \item 违反故障机动车牵引规定的;
    \item 向道路上抛撒物品的。
\end{itemize}

\noindent 机动车载人、载物有下列情形之一的,处一百元罚款:

\begin{itemize}
    \item 非公路客运车辆载人超过核定人数未达到百分之二十的;
    \item 驾驶摩托车违反规定载人的;
    \item 驾驶拖拉机违反规定载人的;
    \item 运载超限的不可解体物品,未按照规定行驶的。
\end{itemize}

\subsection*{罚150}

\noindent 驾驶机动车有下列情形之一的,处一百五十元罚款:

\begin{itemize}
    \item 驾驶安全设施不齐全的车辆上道路行驶的;
    \item 驾驶机件不符合机动车国家安全技术标准的机动车上道路行驶的;
    \item 服用国家管制的精神药品或者麻醉药品后驾驶机动车的;
    \item 患有妨碍安全驾驶机动车的疾病驾驶机动车的;
    \item 使用他人机动车驾驶证的。
\end{itemize}

\subsection*{罚200}

\noindent 驾驶机动车有下列情形之一的,处二百元罚款:

\begin{itemize}
    \item 未悬挂机动车号牌或者未取得机动车临时通行牌证、未按照临时通行牌证载明的有效期限行驶的;
    \item 未按照规定安装号牌的;
    \item 故意遮挡或者污损机动车号牌的;
    \item 改变车身颜色、更换发动机、更换车身或者车架,未在规定的时间内办理变更登记的;
    \item 货运机动车及其挂车的车身或者车厢后未喷涂放大的牌号或者放大的牌号不清晰的;
    \item 大、中型客运机动车未按照规定喷涂核定人数或者经营单位名称的;
    \item 驾驶未按照规定期限进行安全技术检验的机动车的;
    \item 安装、使用影响道路交通安全技术监控设施正常使用的装置或者材料的。
\end{itemize}

\noindent 驾驶机动车有下列情形之一的,处二百元罚款:

\begin{itemize}
    \item \dotuline{逆向}行驶的;
    \item 违反规定在专用车道内行驶的;
    \item 违反交通信号灯指示的;
    \item 违反规定超车的;
    \item 违反规定\dotuline{变更车道}的;
    \item 违反规定\dotuline{会车}的;
    \item 违反规定掉头的;
    \item 违反限制或者禁止通行规定的;
    \item 行经人行横道遇行人通过时,未停车让行的;
    \item 非公路客运车辆载人超过核定人数达到百分之二十以上的;
    \item 货运机动车违反规定附载作业人员的;
    \item 运载爆炸物品、易燃易爆化学品以及剧毒、放射性等危险物品未按照规定行驶的;
    \item 通过铁路道口,违反交通信号或者管理人员指挥的。
\end{itemize}

\noindent 驾驶机动车有下列情形之一的,处二百元罚款:

\begin{itemize}
    \item 拨打接听手持电话、观看电视的;
    \item 下陡坡时故意熄火或者空档滑行的;
    \item 连续驾驶营运车辆超过四个小时,未停车休息或者停车休息时间少于二十分钟的;
    \item 在高速公路和同方向划有二条以上机动车道的道路上行驶,长时间占用超车道的;
    \item 警车、消防车、救护车、工程救险车违反规定使用警报器、标志灯具的;
    \item 违反规定停放车辆,影响其他车辆、行人通行的;
    \item 城市公共汽车违反规定停靠的
\end{itemize}

\noindent 遇前方道路受阻或者前方车辆排队等候、缓慢行驶时,驾驶机动车有下列情形之一的,处二百元罚款:

\begin{itemize}
    \item 违反规定进入路口的;
    \item 违反规定在人行横道或者网状线区域内停车等候的;
    \item 借道超车的;
    \item 占用对面车道的;
    \item 穿插等候车辆的;
    \item 进入非机动车道、人行道行驶的。
\end{itemize}

\noindent 驾驶机动车发生故障或者事故,有下列情形之一的,处二百元罚款:

\begin{itemize}
    \item 未按照规定开启危险报警闪光灯的;
    \item 未按照规定设置警告标志的;
    \item 夜间未开启示廓灯和后位灯的;
    \item 发生交通事故后,未按照规定撤离现场,造成交通堵塞的;
    \item 机动车发生故障后尚能移动,未移至不妨碍交通地点的。
\end{itemize}

\noindent 上道路学习驾驶或者实习期间驾驶机动车,有下列情形之一的,处二百元罚款:

\begin{itemize}
    \item 未按照指定路线、时间学习驾驶或者教练车乘坐无关人员的;
    \item 在实习期间内驾驶禁止驾驶的机动车的。
\end{itemize}

机动车驾驶人在高速公路、城市快速路或者其他封闭的机动车专用道违反道路交通安全法律、法规关于道路通行规定的,处二百元罚款。法律、法规和本条例另有规定的除外。

\subsection*{罚其它金额}

机动车在道路上行驶超过规定时速百分之五十的,处二百元以上二千元以下罚款;机动车在高速公路上行驶,超过规定时速百分之五十的,处一千元以上二千元以下罚款。

在高速公路上停车上下乘客、装卸货物的,对驾驶员处五百元以上二千元以下罚款。

在高速公路上发生事故,不及时报警致使交通堵塞的,对机动车驾驶人处五百元以上一千元以下罚款。

% \vspace{0.5em} % 调整分割线上方间距
% \noindent\textcolor{gray}{\hrulefill} % 自适应长度的灰色实线
% \vspace{0.5em} % 调整分割线下方间距
% 
% 公路客运车辆载客超过核定人数或者违反规定载货的,对机动车驾驶人按下列规定处罚:
% 
% \begin{itemize}
    % \item 超过核定人数未达百分之二十的,处三百元罚款;
    % \item 超过核定人数百分之二十以上未达百分之五十的,处五百元以上一千元以下罚款;
    % \item 超过核定人数百分之五十以上的,处一千五百元罚款;
    % \item 违反规定载货的,处五百元以上一千元以下罚款。
% \end{itemize}
% 
% 货运机动车超过核定载质量或者违反规定载客的,对机动车驾驶人按下列规定处罚:
% 
% \begin{itemize}
    % \item 超过核定载质量未达百分之三十的,处三百元罚款;
    % \item 超过核定载质量百分之三十以上未达百分之五十的,处五百元以上一千元以下罚款;
    % \item 超过核定载质量百分之五十以上的,处一千五百元罚款;
    % \item 货运机动车违反规定载客的,处五百元以上一千元以下罚款。
% \end{itemize}


\begin{table}[htbp]
\centering
\caption{客车超员、货车超载及违规罚款}
    \begin{tabular}{|c|c|c|}
        \hline
        \textbf{罚款}  & \textbf{公路客运车超员} & \textbf{货运机动车超载} \\
        \hline
        300 & $< 20\%$ & $< 30\%$ \\
        \hline
        $500 \sim 1000$ & $20\% \sim 50\%$ & $30\% \sim 50\%$ \\
        \hline
        1500 & $\ge 50\%$ & $\ge 50\%$ \\
        \hline
        $500 \sim 1000$ & 违规载货 & 违规载客 \\
        \hline
    \end{tabular}
    \label{tab:penalty}
\end{table}

\noindent 有下列情形之一的,按以下规定处罚:

\begin{itemize}
    \item 未取得机动车驾驶证、机动车驾驶证被吊销或者被暂扣期间驾驶\uline{营运汽车}的,处一千元以上一千五百元以下罚款;
    \item 未取得机动车驾驶证、机动车驾驶证被吊销或者被暂扣期间驾驶\dotuline{非营运汽车}的,处五百元以上一千元以下罚款;
    \item 将\uline{营运汽车}交由未取得机动车驾驶证、机动车驾驶证被吊销或者被暂扣期间的人驾驶的,处一千元以上一千五百元以下罚款;
    \item 将\dotuline{非营运汽车}交给未取得机动车驾驶证、机动车驾驶证被吊销或者被暂扣期间的人驾驶的,处五百元以上一千元以下罚款。
\end{itemize}

\vspace{0.5em} % 调整分割线上方间距
\noindent\textcolor{gray}{\hrulefill} % 自适应长度的灰色实线
\vspace{0.5em} % 调整分割线下方间距

擅自停用公共停车场(库)或者改变公共停车场(库)用途的,由公安机关交通管理部门责令限期恢复,逾期不恢复的,从停用或者改变用途之日起按每日每平方米三元处以罚款。

擅自设置或者占用、撤销道路临时停车泊位,或者在机动车停车泊位内设置停车障碍的,由公安机关交通管理部门处五百元罚款。